\documentclass[11pt,a4paper]{article}
\usepackage{polski}
\usepackage{indentfirst}
\usepackage{graphicx}
\usepackage{wrapfig}
\usepackage{listings}
\usepackage[cp1250]{inputenc}
\title{Test Driven Development by Example (Beck)}
\author{Wojciech Gawro�ski (afronski@gmail.com)}

\begin{document}
\lstset{language=JavaScript}

\maketitle
\newpage

\tableofcontents
\newpage

\section{Introduction}
Test-Driven Development \textit{mantra} (\textbf{red/green/refactor}):
\begin{itemize}
\item Write test that doesn't work or even fails.
\item Write production code for filling with green this test you've wrote - as simple as it's possible.
\item Refactor - eliminate duplication, fix redundancy.
\end{itemize}
This technique (programming style) in further might be possible to dramatically reduce the defect density of code and make subject of work crystal clear to all involved. If defect density is reduced enough, then QA can shift from reactive work to proactive work and we can have shippable software with new functionality every day, leading to new business relationships with customers. If the number of nasty surprises can be reduced enough the project managers can estimate accurately enough to involve real customers in daily development. Also collaboration will be improved by minute-by-minute talks instead of daily or weekly collaboration.

\section{Motivation}
TDD is a way for managing fear during programming. Not in the bad way - fear in the legitimate (saying "Be careful", "Watch out!" etc.)
\begin{itemize}
\item Fear makes you tentative.
\item Fear makes you want to communicate less.
\item Fear makes you shy away from feedback.
\item Fear makes you grumpy.
\end{itemize}
Instead of these effects we want quick and clear communication, quick and concrete learning and searching helpful feedback. Also we'll have to work on grumpiness on our own.

Tests are ratchets - for working on code we need points which we know it is working, now and forever. Something upon us watches and informs when something goes wrong - by failing a test. Difference between TDD and \textit{Extreme Programming} (which is also a Beck's theorem and method) is subtle. XP says "Here are the things you must be able to do to be prepared to evolve further". TDD is little fuzzier - is awareness of gap between decision and feedback during programming (gap can be changed deliberately). Not all programming tasks can be driven solely by tests - e.g. security, concurrency. Important thing about tests is that these tests must be automated.

\section{TDD cycle}
Most important thing in test-driven development methodology is closed cycle with continous feedback and getting information from test results. We doing as quick and dirty solutions as it's possible for receiving information from test runner. Before writing prodcuction code first we wrote a test, than watching it fails. After that we implement solution simple as possible and we're closing the cycle with code refactoring. We doing that by removing duplication and unnecessary dependencies. And so on, so on - this quick (approx. 10-15 minutes cycles) are providing constant feedback about code and eventually about new bugs, regressions provided by new feature or modifing the old ones. It's all about feedback and small steps.

\section{TDD patterns}
\begin{itemize}
	\item The "no time for testing" death spiral.
	\item Test must be isolated from each other.
	\item Write list of tests which you must to write and stick to the list - even when your code getting ugly, put refactoring tasks on the list.
	\item Test first, not code first.
	\item Assert first - when your write your test try to write assertions first, this is powerful simplyfing techniqe.
	\item Test data - use data which makes tests easy to follow and read.
	\item Evident data - type some of your data explicite (e.g. instead of "pixels - 10" type "pixels - 2 /*margins*/ - 8 /*padding*/" or something more readable).
\end{itemize}

\section{Red bar patterns}
\begin{itemize}
	\item One step test - pick test from list which you're confident to write but not the obvious thing.
	\item Starter test - start with something obvious, a variant of an operation that doesn't do anything. It's all about feedback and small steps - big things will leave you too long without feedback and break TDD cycle.
	\item Explanation test - ask and give explanation in terms of tests.
	\item Learning test - write tests for external libraries or packages before the first time you're going to use a new facility.
	\item Another test - when a tangential idea arises, add a test to the list an go back to the topic. "Offtopic" tests are dangerous because they can get you out of your way.
	\item Regression test - when defect is reported you should write the test first, watch them fail and then fix it and watch it greens with other test cases.
	\item Break - when you feel tired or stuck - take a break!
	\item Do over - when you're feeling lost - throw away code and start over.
\end{itemize}

\section{Testing patterns}
\begin{itemize}
	\item Child test - when your test case turns out to be big - split it ASAP, into smaller test cases that represents the broken part of the bigger test case.
	\item Mock Object - when your test relies on expensive or complicate resource - create a fake version of the resource which responds with prepared constants.
	\item Self Shunt - when you're testing communication object with another object you can pass a test case instead of mocked and faked object for communication test.
	\item Log String - works well with "self shunt" - a simple string field can be usefull for testing observers and other techniques which relies on communication.
	\item Crash Test Dummy - special object which provides error prone situation (e.g. throws an exception instead of doing real work).
	\item Broken test - when you're leaving alone programming session leave the last test broken - easier to dive into context when you're return to work.
	\item Clean Check-in - when you're leaving team programming session leave all tests working and green.
\end{itemize}

\section{Green bar patterns}
\begin{itemize}
	\item Fake-It-'Til-You-Make-It
	\item Triangulate - approximate final algorithms with clear rules.
	\item Obvious implementation - just implement them.
	\item One to many - when you're dealing with collections implement them without collections first then make it work with collections.
\end{itemize}

\section{xUnit patterns}
\begin{itemize}
	\item Assertion
	\item Fixture - common objects need by several tests should be moved into test case instance with overrided setUp method.
	\item External fixture - override tearDown for release external resources.
	\item Test method - name, place and granularity are important.
	\item Exception test
	\item All tests - agregation all suites in one place is convinient.
\end{itemize}

\section{Design patterns}
\begin{itemize}
	\item Command - result of computation is an object (e.g. deffered) with simple set of methods.
	\item Value Object - objects which can be widely shared but identity is unimportant.
	\item Null Object - special case for object usage.
	\item Template Method - invariant sequence of computation with extensible representation (implementing abstract methods in derieved classes).
	\item Pluggable Object - elimination of conditionals duplication - one interface, many implementation and 'pluggable' architecture for specified actions e.g. series of mouse events - down, move, up - on mouse down event you've create object based on pointer position (is it over figure or not) and selection and after that rest of events use this object which handles cases inside implementation (basing on interface).
	\item Pluggable Selector - lightweight but ugly (less than switch statement) selection of method for use. When class diagram is too much you can by reflection (or by name in dynamic language) execute specified (by name) method. Beware, because this use case can spread like a disease.
	\item Factory Method - method instead of explicit constructor call. Flexibility in creation of new objects. In most of implementations is a static method for class which returns newly created object - but level of indirection is enough for flexible object creation. User must remember that this method behaves like constructor but it hasn't got constructor syntax. Use this pattern only when need (otherwise constructor is a better choice).
	\item Imposter - this trick is more than design pattern - it's more like a usage. Explains introduction a new variation of object with existing protocol with different implementation (e.g. collection of objects can behave like an object itself, or null object pattern - absence of object is treated like an actual presence).
	\item Composite - implementation of object which behavior is composition of list of other objects.
	\item Collecting Parameter - object which collects results of operation an spread them over several objects.
\end{itemize}

Singletons are in fact a global variable so you should omit this pattern from your design and rethink your architecture if you need it.

\section{Refactoring}

Process which in TDD maters more than in other methodologies. Indicates smells and pieces of invalid design, redundancy and uncessary complexity. Below it's a list most common refactoring techniques used when you're coding with agreement with TDD methodology:

\begin{itemize}
	\item Isolate Change - Isolate part of object or method which you want to change.
	\item Migrate Data - Temporary duplication of data when you're moving from one representation to another.
	\item Extract Method
	\item Inline Method
	\item Extract Interface
	\item Move Method
	\item Method Object - representation a complicated method which requires several parameters and local variables.
	\item Add Parameter
	\item Convert method parameter to constructor parameter - removing common parameters for API methods to the local class variable and into constructor assignment.
\end{itemize}

\section{Mastering TDD}

Q: How large should me steps be?
A: Smaller steps are better. Small step indicates small amount code to refactor, bigger steps which indicates hundreds lines of code are expensive and time consuming for further improvements.

Q: What don't you have to test?
A: You should write your tests until your fear transform into boredom. In other words - it's a feedback loop but you must answer yourself. At least you should test conditionals, loops, atomic operations and polymorphism. Other - it's all about fear and your answers.

Q: How do you know if you have good tests?
A: Tests are canary in coal mine revealing problems and evil practices by parameters and its behavior. Below is simple list for it:
\begin{itemize}
	\item Long setup code - hundreds lines of code for setting up an object indicates \textit{too-big-object} problem.
	\item Setup duplication - if you can't find common place for common setup code, than there are too many objects too tightly coupled with itself.
	\item Long running tests - if tests are running a long time it's dangerous to run them too few times and in consequence these tests are propably don't work at all.
	\item Fragile test - when tests are failing unexpectedly suggest tha t one part of application is suprisingly affecting to another - break this connection as soon as possible.
\end{itemize}

Q: How does TDD lead to frameworks?

A: Fact -> TDD supports OCP (\textit{Open/Close principle}) - Objects should be open to use and closed for modifications. In fact - by not considering the future code you make code much more likely adaptable in future and more resistant to API changes.

Q: How much feedback do you need?

A: You have to decide (basing on experience and reflection) how many test to write, but in general - you should drive tests amount by your fear and pragmatism, confidence. 

Q: When should you delete test?

A: If two tests are redundant in code path and reader scenario, than one of them should be deleted. Otherwise you should leave it.

Q: How do the programming language and environment influence TDD?

A: Affects but it's all about experience - you cannot be influenced by language and environment because it tempting to make bigger and more complicated steps with no respect to TDD cycle (\textit{Red/Green/Refactor}).

Q: Can you test drive enormous systems?

A: Yes, TDD scales in large systems. But again it depends more on tests quality and people education and skills than project size.

Q: Can you drive development with application-level tests?

A: You shouldn't do that. Application level tests aren't truly unit tests, risk of implementing not what user wants is too high and method is expensive too.

Q: How do you switch to TDD midstream?

A: You shouldn't write tests for old code, but for new fragments and new elements should be driven by tests development. But for elements which are common to both parts we must use a dedicated solution - first we need to break deadlock between missing feedback (tests) and refactoring. We can refactor older code but we're getting feedback from QA or pair programming colleague. Than, after refactoring we can write for whole changed subsystem unit tests (and old parts rewritten looks like test driven).

Q: Who is TDD intended for?

A: For everyone. 

Q: Is TDD sensitive to initial condition?

A: No, natural rythm for TDD (\textit{Red/Green/Refactor}) enforces incremental, small changes and fixes. So in mostly the same way by removing duplication you receive result (nice architecture, confident and pragmatic code).

Q: How does TDD relate to patterns?

A: We don't need strict algorithms and steps for mechanical implementation. In fact whole TDD patterns should be easy to understand and mechanically interpreted by developer (this is far more important than strict steps). Another clue is more surprising - you should think about what system should do than design sort itself out later, after refactoring steps.

Q: Why does TDD work?

A: By Reducing defects, short feedback loop and quick, small decisions. Another case is attracting limit function not as an absolute value. If you write a big set of tests which covers every feature, than simplify code and than add new features and all tests (old and new) are passed than you've create something called "attractor" which round continous function to very big, checked limit domain.

Q: What's with the name?

A: TDD isn't a testing technique - it's an analysis, design, structuring technique of development. 

\begin{itemize}
	\item Test - automated, reified, concrete tests.
	\item Driven - test-first programming, directed on tests.
	\item Development - it's old fashioned name element.
\end{itemize}

Q: How does TDD relate to the practices of XP? 

A: TDD and XP both enhance themselves.

\begin{itemize}
	\item Pairing - tests are excellent conversation between developers pair.
	\item Work fresh - like with XP, TDD advises you to work and write only when you're fresh.
	\item CI - tests are excellent feedback resource running on CI with 15-30 minutes cycle.
	\item Simple design - KISS and YAGNI are most important abbreviations and elements of TDD so it also applies to the XP.
	\item Refactoring - like in XP methodology we're improving code quality for further enhancement (we're applying pressure to the code not documentation - \textit{agile manifesto})
	\item Continous delivery - TDD tests really improves MTBF of your system (\textit{Mean Time Between Failures}).
\end{itemize}

\end{document}
